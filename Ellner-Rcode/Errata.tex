% !TEX TS-program = pdflatex
% The line above tells TeXShop to use the latex -> dvi -> distiller 
% path to produce the pdf. 

\documentclass[12pt]{article}

\usepackage{amsmath,amssymb,amsfonts}
\usepackage{courier,type1cm,array}
\usepackage{makeidx,graphicx,multicol}
\usepackage[bottom]{footmisc} % places footnotes at page bottom
\usepackage{natbib,float}
\usepackage[vmargin=1in,hmargin=1in]{geometry}


% allows compilation with pdflatex, under Windows 
\usepackage{epstopdf}

% Coloring of R code listings
\usepackage[formats]{listings}
\usepackage{color}
\definecolor{mygreen}{rgb}{0.1,0.5,0.1}
\definecolor{mygray}{rgb}{0.5,0.5,0.5}
\definecolor{mymauve}{rgb}{0.58,0,0.82}
\definecolor{mygrey}{rgb}{0.3,0.3,0.1}
\lstset{
language=R,
otherkeywords={data.frame},
basicstyle=\normalsize\ttfamily, 
commentstyle=\normalsize\ttfamily,
keywordstyle=\normalsize\ttfamily,
stringstyle=\color{mymauve}, 
commentstyle=\color{mygreen},
keywordstyle=\color{blue},
showstringspaces=false, xleftmargin=2.5ex,
columns=flexible,
literate={~}{{$\sim \; \; $}}1,
alsodigit={\.,\_},
deletekeywords={on,by,data,R,Q,mean,var,sd,log,family,na,options,q,weights,effects,matrix,nrow,ncol,wt,fix,distance},
}
\lstset{escapeinside={(*}{*)}} 

\lstdefineformat{Rpretty}{
	; = \space,
	\, = [\ \,\]]\string\space,
	<- = [\ ]\space\string\space,
	\= = [\ ]\space\string\space}

\usepackage[compact]{titlesec} 
\usepackage{enumitem}
\setlist{leftmargin=0.75cm}

\usepackage{array}
\newcolumntype{L}[1]{>{\raggedright\let\newline\\\arraybackslash\hspace{0pt}}m{#1}}
\newcolumntype{C}[1]{>{\centering\let\newline\\\arraybackslash\hspace{0pt}}m{#1}}
\newcolumntype{R}[1]{>{\raggedleft\let\newline\\\arraybackslash\hspace{0pt}}m{#1}}

% reduce white space above and below verbatim text blocks
\usepackage{etoolbox}
\makeatletter
\preto{\@verbatim}{\topsep=0.5pt \partopsep=0.5pt }
\makeatother

% let figs,boxes etc. occupy most of a page near where they
% appear in the source file, instead of getting pushed to
% the chapter's end. 
\renewcommand{\topfraction}{0.9}	
\renewcommand{\bottomfraction}{0.9}	
\renewcommand{\textfraction}{0.07}	
\renewcommand{\floatpagefraction}{0.85}	
% floatpagefraction MUST be less than topfraction
\setcounter{topnumber}{2}
\setcounter{bottomnumber}{2}
\setcounter{totalnumber}{4} 	

% Define Box environment for numbered boxes. 
\newcounter{box}
\newcommand{\boxnumber}{\addtocounter{box}{1} \thebox \thinspace}

\floatstyle{boxed}
\newfloat{Box}{tcph}{box}[section]
\numberwithin{Box}{section}

% A few shortcuts
% Define \z so that \z_0 and z'_0 have the same kerning for _0. 
\def\z{z^{}}
\def\Z{\mathbf{Z}}
\def\N{\mathbf{N}}

\newcommand{\be}{\begin{equation}}
\newcommand{\ee}{\end{equation}}
\newcommand{\ba}{\begin{equation} \begin{aligned}}
\newcommand{\ea}{\end{aligned} \end{equation}}

\makeindex

\usepackage[scaled=0.9]{zi4}
\usepackage{textcomp}


\linespread{1.2}

%%%%%%%%%%%%%%%%%%%%%%%%%%%%%%%%%%%%%%%%%%%%%%%%%%%%%%%%%%%%%%%%%%%%%
%%%%% End of the preamble 
 
\begin{document}

\author{Stephen P. Ellner, Dylan Z. Childs and Mark Rees}
\title{Errata for \\ \emph{Data-driven Modeling of Structured Populations: 
A Practical Guide to the Integral Projection Model}} 

\date{Last update: \today} 

\maketitle

\paragraph{Page 12:} We think that the statement about ``piecewise continuous'' in the footnote is true, 
but at least one proof in the book isn't valid regardless of what the curves are that divide $\Z^2$ into 
subregions. A slightly less general definition, which should be sufficient for any applications, is as follows. 
For the basic model where the individual-level state space $\Z$ is a 
bounded interval $[L,U]$, a partition of $\Z$ is a set of 
intervals 
\begin{equation}
\Z_1 = (z_0,z_1), \Z_2 = (z_1,z_2), \cdots, \Z_m = (z_{m-1},z_m) 
\end{equation}
where 
\be
L=z_0 < z_1 < z_2 < \cdots < z_M=U. 
\ee
A partition breaks $\Z^2$ into a set of open rectangles 
$$\Z_{ij} = \Z_i \times \Z_j = \{(z',z): z' \in \Z_i, z \in \Z_j\}.$$
Define the kernels $K_{ij}$ to be $K$ restricted to $\Z_{ij}$. 

We say that $K$ is \emph{piecewise continuous} if there exists a 
partition such that each of the kernels $K_{ij}$ is continuous on $\Z_{ij}$, 
and can be defined on the boundary of $\Z_{ij}$ so that it is continuous
on the closed rectangle $\bar{\Z}_{ij}$ consisting of $\Z_{ij}$ and its boundary.  

The reason this definition works is the general theory in section 6.9
(originally in the Appendices to Ellner \& Rees 2006), applied to the  
closed intervals $\bar{\Z}_{i} =  [z_0,z_1]$ as a set of continuous components, 
with continuous component-to-component kernels $K_{ij}$. In terms of the general     
theory, this is an IPM with continuous kernels. Formally, there is a separate state distribution 
function $n_i(z,t)$ defined on each $\bar{\Z}_i$. But we can also think of there being a 
single distribution function $n(z,t)$ on all of $[L,U]$, consisting of $n_1,n_2, \cdots n_M$ 
side-by-side. 

Each of the boundary points $z_i$ is in two adjacent components, but this doesn't matter because  
single points contribute nothing to the subsequent population. 
As an example consider  $\Z = [0,2]$ and the (contrived) kernel $K(z',z) = 1$ 
if $z'>1$, and 0 otherwise. The partition is $\Z_1 = (0,1), \Z_2 = (1,2)$ and the kernels are 
\be
K_{11}=K_{12}\equiv 0, K_{21}=K_{22}\equiv 1.
\ee
The population dynamics are 
\ba
n_1(z',t+1) & = \int_0^1 K_{11}(z',z) n_1(z,t) dz + \int_1^2 K_{12}(z',z) n_2(z,t) dz  = 0 \\
n_2(z',t+1) & = \int_0^1 K_{21}(z',z) n_1(z,t) dz + \int_1^2 K_{22}(z',z) n_2(z,t) dz  \\
& =  \int_0^1 n_1(z,t) dz + \int_1^2 n_2(z,t) dz.  
\label{eqn:badK1}
\ea
So $n_1(1,t+1)=0$, while $n_2(1,t+1)>0$ unless there were no individuals at time $t$. 
However, the values of $n_1$ and $n_2$ at the one point $z=1$ have no effect on the
integrals in the population dynamics, so we can regard $n(1,t+1)$ as being undefined,
or give it an arbitrary value such the average $n_1(1,t+1)$ and $n_2(1,t+1)$. 

The contrived kernel \eqref{eqn:badK1} is a counter-example to the claim in section 6.9 of the book 
that a piecewise continuous kernel, as defined in that section, maps $L_1(\Z)$ into $C(\Z)$. 
The gap in the proof is the assertion that the functions $f_n$ converge almost everywhere, 
which is not necessarily true regardless of how the partitioning into sets $\mathcal{U}_k$ is done. But on the 
set of domains $\bar{\Z}_{i}$ the component kernels are all continuous and the $f_n$ converge 
pointwise, which is sufficient for the rest of the proof. 
In example \eqref{eqn:badK1}, $n_1$ is continuous on $\bar{\Z}_1$ and 
$n_2$ is continuous on $\bar{\Z}_2$, and that is exactly what it means to be continuous
on the state space with domains $\bar{\Z}_1$ and $\bar{\Z}_2$.  

Please see the related correction below about \textbf{Page 181}. 

\paragraph{Page 15:} Equation (2.3.6), $C_1$ should be $C_0$ as it is in the life-cycle diagram, Figure 2.2. 

\paragraph{Page 29:} 9 lines from the bottom, ``the the'' $\to$ ``the''

\paragraph{Page 119:} \emph{Artemisia} is the correct spelling of the genus. 

\paragraph{Page 181:} As we wrote regarding \textbf{Page 12}, ``piecewise continuous'' needs 
to be defined more narrowly, so that all the claims in this section are valid. 
Even if we ignore the fact that ``continuous curve'' was conveniently undefined, 
eqn. \eqref{eqn:badK1} remains a counterexample to the claim that $K$ maps $L_1(\Z)$ into $C(\Z)$. 

Fortunately, the same approach works in the general setting. A definition of ``piecewise continuous''
that actually works is (informally): we can break up $\Z$ into pieces $\Z_j$, and thereby break up the kernel into
pieces $K_{ij}$ for transitions from $\Z_j$ to $\Z_i$, in such a way that each $K_{ij}$ is continuous. 

Reading the rest of this item requires some familiarity with the basics of measure theory 
and functional analysis. Sorry, there's no way around this. 

We are considering an IPM where the individual state-space $\Z$ is a compact metric space with metric $d$, 
and the kernels are defined relative to a finite Borel measure $\mu$ on $\Z$. 
The population dynamics are 
\be
n(z',t+1) = \int_{\Z} K(z',z)n(z,t) d\mu(z);
\ee 
in the book this is eqn. (6.9.1). The kernel is a function on $\Z^2 = \Z times \Z$, which we give the
$L_2$ product metric and resulting product topology. Subsets of $\Z$ and $\Z^2$ are given the relative
metric and measure. 

Following the definition for the basic model, a partition of $\Z$ is defined to be a collection of disjoint 
open (and therefore measurable) sets $\Z_1, \Z_2, \cdots, \Z_m $ whose boundaries all have $\mu$-measure 0, 
such that the union of their closures $\bar{\Z}_j$ is $\Z$. Again, define the kernels $K_{ij}$ to be $K$ 
restricted to $\Z_{ij} = \Z_i \times \Z_j$. 

We say that a kernel $K$ is piecewise continuous if there is a partition such that each $K_{ij}$ is 
continuous on each $\Z_{ij} = \Z_i \times \Z_j$ and can be defined on the boundary 
of $\Z_{ij}$ so as to be continuous on its closure $\bar{\Z}_{ij}$. At a point $(z',z)$ that is on the boundary
of $\Z_{ij}$ and of $\Z_{kl}$, the values chosen for $K_{ij}(z',z)$ and for $K_{kl}(z',z)$ to ensure continuity
of $K_{ij}$ and $K_{kl}$ do not need to be the same.  

The reason this works is (again) that the set of kernels $K_{ij}$ (extended to be continuous on their
domain $\bar{\Z}_{ij}$) defines a general IPM in the sense of section 6.9 in which the overall 
kernel is continuous. The individual-level state space for the general IPM is the collection of 
sets $\bar{\Z}_j, j=1,2,\cdots m$ considered as distinct sets. Formally, define $\bar{\Z}^*_j$ to be 
the set of ordered pairs $(z,j)$ where $z \in \bar{\Z}_j$, and let $\Z^*$ be the collection 
of all $\bar{\Z}^*_j$ with the metric
\be
d^*((z_1,i),(z_2,j)) = \sqrt{d(z_1,z_2)^2 + (i-j)^2}. 
\ee
The overall kernel is
\be
K^*((z',i),(z,j)) = K_{ij}(z',z). 
\ee
A convergent sequence $x^*_n \to x^*_0$ in $\Z^* \times \Z^*$ must eventually consist of points whose $z$ coordinates 
all lie in the same $\bar{\Z}_{ij}$, so the sequence is convergent in $\bar{\Z}_{ij}$. So for sufficiently large $n$, 
$K^*(x^*_n) = K_{ij}(x^*_n)$ which converges to $K_{ij}(x^*_0)=K^*(x^*_0)$ because $K_{ij}$ is continuous. The overall
kernel is therefore continuous. All results in section 6.9 for continuous kernels therefore also hold for
piecewise continuous kernels under the definition of piecewise continuity given here. 

\paragraph{Page 313:} 2nd line after eqn. (10.7.2), $Cov(Y_1,Y_2)$ $\to$ $Cov[X_2,Y_2]$

\paragraph{Page 313:} line before eqn. (10.7.5) should refer to eqn. (10.7.4)

\paragraph{Page 314:} penultimate line, al\'{a} $\to$ \`{a}  la 




\end{document} 





